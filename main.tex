\documentclass[oneside, a4paper]{article}

\usepackage[T1,T2A]{fontenc}
\usepackage[utf8]{inputenc}
\usepackage[english,russian]{babel}
\usepackage{url, hyperref}
\usepackage[shortlabels]{enumitem}
\usepackage{amssymb,amsthm,amsmath,mathtools}
\usepackage{listings}

\usepackage{tocbibind}

\usepackage{thmtools}
\usepackage{thm-restate}

\usepackage{geometry}

% \geometry{
%    a4paper,
%    left=20mm,
%    right=20mm,
%  }

\newtheorem{theorem}{Теорема}
\newtheorem{corollary}{Следствие}
\newtheorem{proposition}{Предложение}
\newtheorem{lemma}{Лемма}
\theoremstyle{definition}
\newtheorem{definition}{Определение}
\theoremstyle{remark}
\newtheorem{remark}{Замечание}

\renewcommand{\thesection}{}
\renewcommand{\thesubsection}{}

\newcommand\rowm{\ensuremath{\operatorname{row}^-}}
\newcommand\colm{\ensuremath{\operatorname{col}^-}}

\DeclareMathOperator{\lcm}{LCM}
\DeclareMathOperator*{\argmax}{arg\,max}

\DeclarePairedDelimiter\abs{\lvert}{\rvert}
\DeclarePairedDelimiter\ang{\langle}{\rangle}
\DeclarePairedDelimiter\floor{\lfloor}{\rfloor}
\DeclarePairedDelimiter\ceil{\lceil}{\rceil}

\DeclareRobustCommand{\divby}{%
  \mathrel{\vbox{\baselineskip.65ex\lineskiplimit0pt\hbox{.}\hbox{.}\hbox{.}}}%
}

\begin{document}

\title{}
\author{Олейников Иван}
\date{\today}
\maketitle

% \abstract{В прошлых работах Hiltgen \cite{hiltgen_article, hiltgen} были
% построены несколько слабо надёжных односторонних функций. Слабо надёжная
% одностороннаяя функция "--- это такая функция, обратить которую на определённой
% доле входов как минимум в константу раз труднее, чем вычислить её значение в
% точке. Под трудностью здесь понимается размер минимальной схемы из произвольных
% двоичных гейтов.
% 
% В этом документе предлагается определение слабо надёжного трудного бита.
% Неформально говоря, трудным битом для биективной функции $f$ здесь называется
% такая функция $h$, что вычислить $h(x)$ по заданному $f(x)$ (то есть вычислить
% $h \circ f^{-1}$) как минимум в константу раз труднее, чем вычислить $f(x)$ или
% $h(x)$ по заданному $x$.
% 
% Кроме того, в этому документе приводятся конструкции трудных битов для нескольки
% линейных функций, основанных на односторонних функциях из работ Hiltgen.}

\tableofcontents

\section{Введение}



\section{Определения}

\begin{itemize}
\item $\rowm_i(M)$ "--- функция, удаляющая $i$-тую строку из матрицы $M$;
\item $\colm_j(M)$ "--- аналогично, функция, удаляющая $j$-тый столбец из матрицы $M$;
\item $e_i$ "--- вектор, имеющий единицу на позиции $i$ (размерность вектора
должна быть ясна из контекста);
\item \dots.
\end{itemize}

Символом $D_l$ обозначим квадратную матрицу $l \times l$, у которой на главной
диагонали и ниже неё стоят нули, а на всех остальных позициях "--- единицы.

\begin{definition}
Для последовательности $l_1 \dots l_k$ символом $A_{l_1, \dots l_k}$ обозначим
квадратную матрицу размера $(1 + \sum l_i) \times (1 + \sum l_i)$. В левом-верхнем
углу этой матрицы будет стоять единица. За ней вдоль главной диагонали будут идти
матрицы $D_{l_1}, D_{l_2} \dots D_{l_k}$ (как в блочно-диагональной матрице) до
правого нижнего угла. На всех остальных позициях, не занятых матрицами $D_{l_i}$
и левой верхней единицей, будут стоять единицы.
\end{definition}

Для удобства записи, мы будем иногда обозначать последовательность из нижнего
индекса $A_L$ вектором $L = (l_1 \dots l_k)$.

\begin{remark}
Строки и столбцы матрицы $A_L$ можно переставить таким образом, чтобы она
оказалась симметричной относительно главной диагонали.
\end{remark}

К примеру, $A_{1, 2, 3}$ будет выглядеть как
\[
\begin{bmatrix}
1 & 1 & 1 & 1 & 1 & 1 & 1 \\
1 & \mathbf{0} & 1 & 1 & 1 & 1 & 1 \\
1 & 1 & \mathbf{0} & \mathbf{1} & 1 & 1 & 1 \\
1 & 1 & \mathbf{0} & \mathbf{0} & 1 & 1 & 1 \\
1 & 1 & 1 & 1 & \mathbf{0} & \mathbf{1} & \mathbf{1} \\
1 & 1 & 1 & 1 & \mathbf{0} & \mathbf{0} & \mathbf{1} \\
1 & 1 & 1 & 1 & \mathbf{0} & \mathbf{0} & \mathbf{0} \\
\end{bmatrix},
\]
где жирным выделены подматрицы $D_1$, $D_2$, $D_3$.

\section{Доказательство}

\begin{proposition} \label{easy}
Для любой строки $i$ (столбца) матрицы $A_L$ существует такой столбец $j$
(строка), что удаление этих строки и столбца из $A_L$ даст $A_{L - e_r}$ "---
другими словами, уменьшит на единицу одну из компонент $l_r$ вектора $L$:
\[
(\rowm_i \circ \colm_j)(A_L) = A_{L - e_r}.
\]

При этом, если после уменьшения компоненты $l_r$ на единицу она стала равна нулю,
то будем считать, что она совсем удаляется из вектора $L$.
\end{proposition}
\begin{proof}
Если $i = 1$ и была удалена первая строка, состоящая из всех единиц, тогда
установим $j := 2$ и удалим второй столбец. После этого появится новая состоящая
из единиц строка, и $l_1$ уменьшится на единицу.

Если $i > 1$ и была удалена строка, пересекающая какую-то из подматриц $D_r$,
тогда установим $j := i$ и удалим симметричный столбец. Легко проверить, что
\[
(\rowm_{i'} \circ \colm_{i'})(D_l) = D_{l-1}
\]
для любого $i' \leq l$.
\end{proof}

\begin{proposition} \label{tough}
Для вектора $L$ обозначим $n = 1 + \sum l_i$.

Пусть $n \geq 2$, а функция
\[
f_n : \{0, 1\}^n \times \{0, 1\}^n \to \{0, 1\} \hspace{2.5cm}
f_n(x, y) = x^\top A_L y + u^\top x + v^\top y
\]
для некоторых $n$-мерных векторов $u$ и $v$. А $g : \{0, 1\}^2 \to \{0, 1\}$
"--- произвольная функция от двух бит, которая зависит от обоих своих входов.

Тогда никакая функция, корректно вычисляющая $f$ более чем на $7/8$ входов, не
является функцией от $g$ двух своих входных бит.
\end{proposition}

\begin{proof}
Пусть функция $h(x, y)$ корректно вычисляет $f(x, y)$ на более, чем $7/8$ доле
всех $\{0, 1\}^{2n}$ входов.

Обозначим для удобства элементы матрицы $A_L$ как $\{a_{i, j}\}_{(i, j) \in {[n]^2}}$.

Рассмотрим два любых входных бита функции $f(x, y)$. Рассмотрим несколько
возможных случаев того, каким из входных векторов $x$ и $y$ функции $f$ они
принадлежат.
\begin{description}
\item{Биты "--- это $x_i$ и $x_j$.} У строк $i$ и $j$ матрицы $A_L$ обязательно
найдётся позиция (столбец), в которой они отличаются. Обозначим номер такого
столбца как $\delta$, а соответствующую столбцу переменную "--- как $y_\delta$.
Не теряя общности, будем считать, что в позиции $\delta$ строка $i$ имеет ноль,
а строка $j$ "--- единицу: $a_{i, \delta} = 0, a_{j, \delta} = 1$.

Ясно, что найдётся какая-то подстановка $\rho$, которая фиксирует все входные
переменные $f$, кроме $x_i, x_j, y_\delta$ так, что при этом $h \rvert _ \rho$
корректно вычисляет $f \rvert _ \rho$ на более чем $7/8$ доле входов $\{0,
1\}^3$. Покажем, что $h \rvert _ \rho$ не может быть функцией от $g$ двух своих
входных битов.

Поймём, какой вид будет иметь $f \rvert _ \rho$. 
\[
f(x, y) = \sum_{i', j'} a_{i', j'} x_{i'} y_{j'} + \sum_{i'} u_{i'} x_{i'} + \sum_{j'} v_{j'} y_{j'}.
\]
Если зафиксировать $\rho$, то получится (помним, что $a_{i, \delta} = 0, a_{j, \delta} = 1$)
\[
f \rvert _ \rho (x_i, x_j, y_\delta) = x_j y_\delta + x_i b_i + x_j b_j + y_\delta c_\delta + d,
\]
где $b_i, b_j, c_\delta$ и $d$ "--- это константы, которые получаются от
фиксирования входов $f$.

Уже сейчас можно понять, что такую функцию $f \rvert _ \rho$ нельзя вычислить на
более чем $7/8$ входов функцией от $g(x_i, x_j)$ ни при каких значениях констант
$b_i$, $b_j$, $c_\delta$ и $d$.

% Но убедимся в этом по-честному.
% Не теряя общности, можно считать, что константа $d = 0$.

\item{Биты "--- это $y_i$ и $y_j$.} Этот случай симметричен предыдущему.
\item{Биты "--- это $x_i$ и $y_j$.} Так как $A_L$ имеет одну строку и один
столбец, полностью состоящие из единиц, либо в строке $i$, либо в столбце $j$
найдётся единица на позиции, отличной от $(i, j)$. Не теряя общности будем
считать, что такая единица нашлась в столбце $j$ и расположена в клетке $(i',
j)$.

Аналогично предыдущему пункту выберем такую подстановку $\rho$ всех переменных,
кроме $x_i, x_{i'}, y_j$, чтобы $f \rvert _ \rho$ и $h \rvert _ \rho$ совпадали
не менее, чем на $7/8$ входов.

Обозначив для краткости $a := a_{i, j}$, получим следующее выражение для $f
\rvert _ \rho$.
\[
f \rvert _ \rho (x_i, x_{i'}, y_j) = a x_i y_j + x_{i'} y_j + x_i b_i + x_{i'} b_{i'} + y_j c_j + d,
\]
где $b_{i}, b_{i'}, c_j$ и $d$ "--- константы.

Легко проверить вручную, что такая никакая функция, вычисляющая эту верно более
чем на $7/8$ доле входов, не может быть функцией от $g(x_i, y_j)$ ни при каких
значениях констант $b_{i}, b_{i'}, c_j$ и $d$.
\end{description}

(В том, что утвеждения из пунктов выше выполняются для $f \rvert _ \rho$, можно
убедиться вручную. У меня это получилось сделать только разборов случаев по
значениям констант. Либо можно перебрать программой все значения констант.)
\end{proof}

\begin{proposition} \label{single_one}
Пусть матрица $M$ имеет единицу в клетке $(i, j)$, а $u$ и $v$ "--- произвольные
вектора.

Тогда любая функция, вычисляющая $f(x, y) = x^\top A y + u^\top x + v^\top y$ на
$> 7/8$ доле входов, обязана зависеть от $x_i$.
\end{proposition}
\begin{proof}
Заметим, что $f_{x_i = 0, y_j = 0}$, $f_{x_i = 0, y_j = 1}$ и $f_{x_i = 1, y_j =
0}$ равны друг другу. А $f_{x_i = 1, y_j = 1}$ не равно ни одному из этих значений.
Поэтому любая функция, которая не зависит от $x_i$ обязана отличаться от $f$ на
$1/4$ доле входов.
\end{proof}

\begin{lemma} \label{main}
При $n = 1 + \sum l_i \geq 2$ для любых $u, v \in \{0, 1\}^n$ функция
\[
f_n(x, y) = x^\top A_L y + u^\top x + v^\top y
\]
не может быть вычислена на более чем $7/8$ входов схемой размера менее $3n -
\max \{l_i\} - 5$. То же самое другими словами:
\[
C_{7/8}(f_n) \geq 3n - \max \{l_i\} - 5.
\]
\end{lemma}
\begin{proof}
Индукция по $n$.

\emph{База $n = 2$}. Очевидно, $C_{7/8}(f_n) \geq 0$.

\emph{Переход $n > 2$.} Рассмотрим любую функцию $h_n$, которая корректно
вычисляет $f_n$ на более чем $7/8$ доле входов. Рассмотрим схему минимального
размера для $h_n$. Обозначим первый гейт этой схемы в порядке топологической
сортировки как $g$. На вход $g$ подаются два входа схемы. По предложению \ref{tough}
один из этих входов должен подаваться на какой-то ещё гейт схемы. Зафиксируем
этот вход схемы $z_1$ в значение $b_1$, удалив из схемы два гейта. Если $h_n$
вычисляла $f_n$ на $> 7/8$ доле входов, то можно выбрать $b_1$ так, чтобы
$h_n \rvert_{z_1 = b_1}$ вычисляла $f_n \rvert_{z_1 = b_1}$ на $ > 7/8$ доле входов.

Такое фиксирование удалит либо одну строку, либо столбец из матрицы $A_L$. Не
теряя общности, будем считать, что была удалена строка. Обозначим её номер как
$i$. По предложению \ref{easy} найдётся столбец $j$, такой что после удаления
строки $i$ и столбца $j$ из матрицы $A_L$ получится $A_{L'}$, где $L' = L - e_r$
для некоторого $r$.

Для удаления столбца $j$ из матрицы, полученной после фиксирования $z_1$,
зафиксируем соответствующий ему входной бит. Обозначим этот бит как $z_2$, а
значение, в которое его зафиксируем, как $b_2$. Значение $b_2$ можно выбрать
таким образом, чтобы $h_n \rvert _ {z_1 = b_1, z_2 = b_2}$ верно вычисляла
$f_n \rvert _ {z_1 = b_1, z_2 = b_2}$ на $> 7/8$ входов.

Напомним, что $L = (l_1, \dots l_k)$. Рассмотрим два случая.
\begin{itemize}
\item Если $k \geq 2$, то после удаления строки $i$ в каждом столбце останется
как минимум одна единица. Поэтому из предложения \ref{single_one} следует, что
$h \rvert _ {z_1 = b_1}$ зависит от $z_2$. А это значит, что при фиксировании
$z_2 = b_2$ из схемы удалится как минимум один гейт. В этом случае, мы удалили
три гейта из схемы, поэтому $C(h \rvert _ {z_1 = b_1, z_2 = b_2}) \leq C(h) - 3$
и уменьшили одно из значений $l_i$ на единицу. По индукционному предположению
\[
\begin{aligned}
C(h \rvert _ {z_1 = b_1, z_2 = b_2}) &\geq 3(n - 1) - \underbrace{\max \{l_i'\}}_{\leq \max \{l_i\}} - 5 \\
&\geq 3(n - 1) - \max \{l_i\} - 5.
\end{aligned}
\]
Утверждение доказано.

\item Если $k = 1$. В этом случае после фиксирования $z_2 = b_2$ из схемы могло
ничего и не удалиться. Но всё равно мы удалили из схемы как минимум два гейта
фиксированием $z_1 = b_1$. Поэтому аналогично предыдущему пункту верно $C(h
\rvert _ {z_1 = b_1, z_2 = b_2}) \leq C(h) - 2$, к тому же по индукционному
предложению
\[
\begin{aligned}
C(h \rvert _ {z_1 = b_1, z_2 = b_2}) &\geq 3(n - 1) - \underbrace{\max \{l_i'\}}_{\max \{l_i\} - 1} - 5 \\
&= 3(n - 1) - \max \{l_i\} + 1 - 5.
\end{aligned}
\]

Утверждение доказано.
\end{itemize}
\end{proof}

\subsection{Следствия и конструкции}
Обозначим символом $M_{k, n}$ матрицу $A_L$, где вектор $L = \{l_1 \dots l_k\}$.
Сумма компонент $L$ равна $n - 1 = \sum l_i$, а значения компонент отличаются
максимум на единицу: $l_i \in \left \{ \floor{\frac {n - 1} k}, \ceil{\frac {n -
1} k} \right\}$.

Следующее является следствие из леммы \ref{main}.

\begin{corollary}
Для $f_{n, k}(x, y) = x^\top M_{k, n} y$ верно
\[
C_{7/8}(f_{n, k}) \geq 3n - \ceil*{\frac {n - 1} k} - 5.
\]
\end{corollary}

В работе Hiltgen \cite[p.~63]{hiltgen} построена такая односторонняя функция
$\lambda_{n ,k}$ и доказана следующая теорема о её сложности.

\begin{theorem}[Hiltgen]
Пусть $\lambda_{n, k}(x) = M_{n, k}^{-1} x$, а $\lambda^{-1}_{n, k}(x) = M_{n,
k} x$, тогда
\[
n \leq C(\lambda_{n, k}) \leq n + k - 1 \quad \quad C(\lambda_{n, k}^{-1}) = \floor*{ \frac {2k - 1} k (n - 1) }
\]
\end{theorem}

Теперь можно выбрать в качестве односторонней функции $h_{n,k}(x, y) = (x,
\lambda_{n, k}(y))$, а в качестве трудного бита для неё скалярное произведение
$p(x, y) = x^\top y$. Тогда функция $f$ окажется как раз той функцией, которую
должен вычислить атакующий, которому известно значение односторонней функции
$h_{n, k}(x, y)$ и который хочет вычислить $p_(x, y)$: $f_{n, k} = p_n \circ h_{n,
k}^{-1}$.

Теперь, выбирая конкретные значения для $k$, построим несколько односторонних
функций с трудным битом.
\begin{description}
\item[$k = 2$.] В этом случае получается $3/2$-надёжная одностороннаяя функция
из работы \cite{hiltgen}.
\[
\begin{aligned}
C(h_{n, k}) &\leq n + 1 \\
C(h_{n, k}^{-1}) &\geq \frac 3 2 (n - 1) \\
C(f_{n, k}) &\geq (3 - \frac 1 2)n - 6. \\
\end{aligned}
\]
\item[$k = n - 1$.] Это крайний случай, здеь получается, что $h_{n, k}$ так же
легко обратить, как и вычислить, но оценка на $C(f)$ получается самая лучшая:
\[
\begin{aligned}
C(h_{n, k}) &\leq 2n - 2 \\
C(h_{n, k}^{-1}) &\geq 2n - 3 \\
C(f_{n, k}) &\geq 3n - 6. \\
\end{aligned}
\]
\item[$k = \sqrt{n}$.] В этом случае получаются асимптотически наилучшие
соотношения между сложностями функций:
\[
\begin{aligned}
C(h_{n, k}) &\leq n + \sqrt{n} - 1 \\
C(h_{n, k}^{-1}) &\geq 2n - \sqrt n - 3 \\
C(f_{n, k}) &\geq 3n - \sqrt{n} - 6. \\
\end{aligned}
\]
\end{description}
Вычисление трудного бита во всех трёх случаях $C(p_n) = 2n - 1$.

\bibliography{main}{}
\bibliographystyle{plain}

\end{document}
