\documentclass[oneside, a4paper]{article}

\usepackage[T1,T2A]{fontenc}
\usepackage[utf8]{inputenc}
\usepackage[english,russian]{babel}
\usepackage{url, hyperref}
\usepackage[shortlabels]{enumitem}
\usepackage{amssymb,amsthm,amsmath,mathtools}
\usepackage{listings}

\usepackage{savesym}

\savesymbol{C}
\savesymbol{G}
\usepackage{complexity}
\restoresymbol{TXF}{C}
\restoresymbol{TXF}{G}

\usepackage{tikz-cd}

\usepackage{tocbibind}

\usepackage{thmtools}
\usepackage{thm-restate}

\usepackage{geometry}

% \geometry{
%    a4paper,
%    left=20mm,
%    right=20mm,
%  }

\newtheorem{theorem}{Теорема}
\newtheorem{exercise}{Упражнение}
\newtheorem{corollary}{Следствие}
\newtheorem{proposition}{Предложение}
\newtheorem{lemma}{Лемма}
\theoremstyle{definition}
\newtheorem{definition}{Определение}
\theoremstyle{remark}
\newtheorem{remark}{Замечание}

% \renewcommand{\thesection}{}
% \renewcommand{\thesubsection}{}

\newcommand\rowm{\ensuremath{\operatorname{row}^-}}
\newcommand\colm{\ensuremath{\operatorname{col}^-}}
\newcommand\F{\ensuremath{{\mathbb F}_2}}

\DeclareMathOperator{\lcm}{LCM}
\DeclareMathOperator*{\argmax}{arg\,max}
\DeclareMathOperator*{\val}{val}

\DeclarePairedDelimiter\abs{\lvert}{\rvert}
\DeclarePairedDelimiter\ang{\langle}{\rangle}
\DeclarePairedDelimiter\floor{\lfloor}{\rfloor}
\DeclarePairedDelimiter\ceil{\lceil}{\rceil}

\DeclareRobustCommand{\divby}{%
  \mathrel{\vbox{\baselineskip.65ex\lineskiplimit0pt\hbox{.}\hbox{.}\hbox{.}}}%
}

\begin{document}

\title{Конструкции доказуемо надёжных в слабом смысле трудных битов \\ \large Дипломная работа}
\author{\normalsize Выполнил: Олейников Иван \\ \normalsize Научный руководитель: Гирш Эдуард Алексеевич}
\date{\today}
\maketitle

\abstract{В этой работе вводится определение надёжного в слабом смысле тругоно
бита, а также приводятся конструкции таких трудных битов для некоторых линейных
функций.

Неформально говоря, надёжным в слабом смысле трудным битом для биективной
булевой функции $f$ здесь называется такая функция $h$, что вычислить $h(x)$ по
заданному $f(x)$ (то есть вычислить $h \circ f^{-1}$) как минимум в константу
раз труднее, чем вычислить $f(x)$ или $h(x)$ по заданному $x$. Под трудностью
функции здесь понимается размер минимальной вычисляющей её схемы из произвольных
гейтов арности $2$.

Функциями, для которых в этой работе строится надёжный в слабом смысле
трудный бит, являются построеные в прошлых работах А. Хильтгена
\cite{hiltgen1993,hiltgen1994} слабо надёжные односторонние функции.
Слабо надёжная одностороннаяя функция "--- это такая функция, обратить которую
как минимум в константу раз труднее, чем вычислить
её значение в точке.}

\tableofcontents

\section{Введение}

Неформально говоря, односторонеей функцией является такая функция $f$, вычислить
которую в любой точке $x$ легко, но по значению которой $f(x)$ трудно найти
такую точку, в которой функция примет это значение \cite{goldreich}. Это
изображено на рисунке \ref{fig_owf}, пунктирная стрелка обозначает задачу
противника, которая должна быть сложной, а сплошной линией "--- задача честного
пользователя, которая должна быть простой.
\begin{figure}[h]
\[
\shorthandoff{"}
\begin{tikzcd}
x \arrow[rr, bend left, "f"] & \hspace{1cm} & f(x) \arrow[ll, bend left, dashrightarrow, "f^{-1}"] \\
\end{tikzcd}
\shorthandon{"}
\]
\caption{Схема отображений односторонней функции $f$}
\label{fig_owf}
\end{figure}

Односторонние функции являются базовым криптографическим примитивом для
построения построения других, таких как псевдослучайные генераторы, протоколы
привязки \begin{foreignlanguage}{english}(bit commitment)\end{foreignlanguage},
шифрование с секретным ключом и цифровая подпись. В настоящее время неизвестно
ни одной конструкции односторонней функции, для которой была бы безусловно
доказана надёжность. Более того неизвестно даже, существуют ли такие функции.
Что вовсе не удивительно, ведь если доказать существование односторонних
функций, это позволит решить некоторые давно открытые вопросы теории
сложности, к примеру, из их существования будет следовать $\NP \nsubseteq
\BPP$ и, следовательно, $\P \neq \NP$ \cite{goldreich}.

Поэтому надежды на то, что для какой-то функции будет доказано, что она является
одностронней, мало. И все известные утверждения о надёжности односторонних
функций доказываются условно, исходя из некоторых популярных предположений.

Раз не удаётся доказать надёжность ни для какой односторонней функции в
классическом её определении, то быть может её удастся доказать, если выбрать
другое определение с более слабыми требованиями? Именно такой вопрос исследовал
А. Хильтген в своих работах \cite{hiltgen1993,hiltgen1994}.

Он предложил определить надёжную в слабом смысле одностороннюю перестановку
порядка $k$ как перестановку $f_n : \{0, 1\}^n \to \{0, 1\}^n$, обратить
которую примерно в $k$ раз <<сложнее>>, чем вычислить её: $C(f_n^{-1}) \sim k
\cdot C(f_n)$. Сложностью $C(f)$ здесь называется минимальный размер схемы,
вычисляющей функцию $f$. Кроме того, Хильтген построил линейные односторонние
перестановки со слабой надёжностью порядка $3/2$ и $2$, а также нелинейную
порядка $2$.

Предложенно Хильтгеном определение слабой надёжности схоже с классическим
определением в том, что оба определения требуют, чтобы задача противника "---
вычисление $f_n^{-1}$ решалась <<труднее>>, чем задача честного пользователя
"--- вычисление $f_n$. Отличие этих определений в том, что они по-разному
уточняют, что означает <<труднее>>: слабая надёжность требует, чтобы сложности
функций отличались в константу раз; а классическое определение требует, чтобы
эти сложности отличались на суперполиномиальный множитель, и в то же время
противнику разрешается корректно вычислять лишь в среднем случае: противник
должен коррекно вычислять $f_n^{-1}$ доле $1 / \poly(n)$ от всех входов, а на
всех остальных ему позволено ошибаться.

Определения надёжности в слабом смысле для других криптографических примитивов,
предложенные в следующих работах \cite{HN09,DN11,hirsch_milanich_nikolenko},
как и предложенное в этой дипломной работе, схожи с определением Хильтгена для
односторонних функций и точно так же требуют, чтобы задачи противника были в
константу раз труднее задач честных пользователей. Но определения, предложенные
в следующих работах, позволяют противнику ошибаться на некоторой константной
доле входов.

В следующих за Хильтгеном работах о надёжности в слабом смысле предлагались
определения надёжности и конструкции односторонних функций с секретом
(\begin{foreignlanguage}{english}trapdoor function\end{foreignlanguage}). В
работе \cite{HN09} Э. А. Гирш и С. И. Николенко предложили определение надёжной
в слабом смысле односторонней функции с секретом, схожее с предложенным
Хильтгеном. Они также привели такую конструкцию с надёжностью порядка $25/22$.
Затем \cite{DN11} А. П. Давыдов и С. И. Николенко привели линейную конструкцию
слабо надёжной односторонней функции с секретом порядка $5/4$. После этого Э. А.
Гирш, О. Ю. Меланич и С. И. Николенко \cite{hirsch_milanich_nikolenko} привели
нелинейную конструкцию с порядком надёжности $7/5$, а также линейную с порядком
надёжности $25/22$.

В этой дипломной работе предлагается определение надёжного в слабом
смысле трудного бита (\begin{foreignlanguage}{english}hard-core
predicate\end{foreignlanguage}) для функции. Неформально можно определить
трудный бит для функции $f$ как такую функцию $h$, что вычислить $f(x)$ и
$h(x)$ по заданному $x$ легко, но вычислить $h(x)$ по заданному $f(x)$ трудно
\cite{goldreich}. Это иллюстрируется рисунком \ref{fig_hp}, задачи честного
пользователя обозначены сплошными стрелками, а задача противника "---
пунктирной.

\begin{figure}[h]
\[
\shorthandoff{"}
\begin{tikzcd}
x \arrow[rr, bend left, "f"] \arrow[rd, bend right, "h"] & ~ & f(x) \arrow[ld, bend left, dashrightarrow, "h \circ f^{-1}"] \\
~ & h(x) & ~
\end{tikzcd}
\shorthandon{"}
\]
\caption{Схема отображений функции $f$ и её трудного бита $h$}
\label{fig_hp}
\end{figure}

% Что ещё строится?
В классической криптографии трудный бит является промежуточным примитивом,
используя его и одностороннюю функцию строится, к примеру, псевдослучайный
генератор.

Кроме определения, в этой работе также приводится несколько конструкций трудного
бита. Функциями $f$, для которых строятся трудные биты $h$, являются линейные
односторонние функции из работ Хильтгена, в которые были добавлены фиктивные
входы, которые подаются на выход неизменными.

Схожая техника <<раздувания входа>> фиктивными битами была применена в работе
О. Голдрейха и Л. Левина \cite{goldreich_levin} для получения односторонней
перестановки (в классическом определении), имеющей трудный бит, из любой
односторонней перестановки. В их работе доказывается, что если $f' : \{0, 1\}^n
\to \{0, 1\}^n$ "--- односторонняя перестановка, то функция $h(x, y) = \ang{x,
y}$ будет являться трудным битом для односторонеей функции $f(y, x) = (y,
f(x))$. (Символом $\ang{\cdot, \cdot}$ обозначается скалярное произведение
$n$-битных выкторов.) Построенные в этой работе функции с трудными битами
устроены как описанные здесь $f$ и $h$, где в качестве $f'$ используются слабо
надёжные односторонние функции Хильтгена.

Несмотря на схожесть построенной в этой работе конструкции с конструкцией
Голдрейха и Левина, доказательство надёжности (классической и слабой) для них
совсем не похожи. В этой работе для доказательства используется техника удаления
гейтов (\begin{foreignlanguage}{english}gate elimination\end{foreignlanguage}),
применением которой доказаны все известные сейчас нижние оценки на схемную
сложность.

В секции 2 даются все нужные далее определения, включая определение
надёжного в слабом смысле трудного бита и схемной сложности, в секции 3
описывается семейство надёжных в слабом смысле односторонних функций из работы
Хильтгена, приводятся построенные Хильтгеном доказательства их надёжности,
а также описывается связь конструкций Хильтгена с приводимыми в этой работе
конструкциями, затем в секции 4 приводится нижняя оценка на сложность задачи
противника. Наконец в секции 4 подводится итог работы и описываются следствия из
оценок сложности.

\section{Определения}

В этой дипломной работе будет идти речь об односторонних функциях и трудных
битах для них. Поэтому рассмотрим для примера представленное ниже определение
односторонней функции из книги \cite[гл. 2]{goldreich}.

\begin{definition}
Функция $f : \{0, 1\}^n \to \{0, 1\}^{l(n)}$ называется (сильно) односторонней,
если
\begin{enumerate}[(a)]
\item $f$ вычислима полиномиальным по времени алгоритмом;
\item но для любого семейства схем $\{ C_n \}_{n \in \mathbb N}$ полиномиального
от $n$ размера и любого полинома $p(n)$ при достаточно больших $n$ выполняется:
\[
\Pr_{x \gets \{0, 1\}^n}[(f \circ C_n \circ f)(x) = f(x)] < \frac 1 {p(n)}.
\]
\end{enumerate}

Другими словами, вероятность того, что схема-противник $C_n$ сумеет по заданному
$f(x)$ найти какой-нибудь элемент $x'$, на котором $f(x') = f(x)$, должна
убывать быстрее любого обратного полинома $1 / p(n)$.

\emph{(Есть так же вариант определения <<равномерной>> односторонней функции, в
котором вместо семейства схем $\{C_n\}$ противником является полиномиальный по
времени вероятностный алгоритм $A$.)}
\end{definition}


\section{Введение}

В этой работе даётся определение слабо надёжного трудного бита, а также
строятся несколько таких конструкций. Определение содержится в этой секции,
кроме него в ней даётся высокоуровневый обзор того, о чём будут следующие
секции.

\begin{figure}[h]
\[
\shorthandoff{"}
\begin{tikzcd}
x \arrow[rr, "f"] \arrow[rd, "h"] & ~ & f(x) \arrow[ld, dashrightarrow, "h \circ f^{-1}"] \\
~ & h(x) & ~
\end{tikzcd}
\shorthandon{"}
\]
\caption{Схема отображений функции $f$ и её трудного бита $h$}
\end{figure}


Символом $C(f)$ обозначим размер минимальной схемы с гейтами из $\{0, 1\}^2 \to
\{0, 1\}$, вычисляющей функцию $f : \{0, 1\}^n \to \{0, 1\}^m$. А символом
$C_\alpha(f)$ обозначим размер, минимальной схемы, которая вычисляет $f$
корректно на \emph{большей} чем $\alpha$ доле всех входов $\{0, 1\}^n$.

\subsection{Мотивация и определения}

TODO: Добавить ссылки на \cite{hirsch_milanich_nikolenko} и ещё одну-две статьи
Николенко.

TODO: Уместить в нужные места эти определения:

\begin{definition}[Схемная сложность]
Булевой схемой называется ациклический ориентированный граф, вершины которого
бывают двух типов:
\begin{itemize}
\item Входы, имеющие входную степень $0$. Каждая такая вершина $v$ помечена
  переменной $x_v$ из набора $x_1$, $x_2$ \dots $x_n$.
\item Гейты, имеющие входную степень $2$. Каждая такая вершина $v$ помечена
  функцией $\phi_v : \{0, 1\}^2 \to \{0, 1\}$.
\end{itemize}

Если заданы значения переменных $\bar{x} = (x_1$ \dots $x_n)$, то каждый вход
схемы $v$ можно пометить значением соответствующей переменной "--- $\val_{\bar
x}(v) := x_v$. Затем индуктивно распространить эти значения от входов схемы по
всем гейтам схемы таким образом: если в гейт $w$ входят рёбра из вершин $u$ и
$v$, которые уже помечены значениями $\val_{\bar x}(u)$ и $\val_{\bar x}(v)$,
пометим $\val_{\bar x}(w) := f_w(\val_{\bar x}(u), \val_{\bar x}(v))$.

\end{definition}

\begin{definition}[Пренебрежимо малая функция]
Функция $s : \mathbb N \to \mathbb N$ называется пренебрежимо малой, если для
любого полинома $p(n)$ при достаточно больших $n$ значение $s(n)$ меньше $1 /
p(n)$:
\[
\exists N,\quad \forall n > N,\quad s(n) < \frac 1 {p(n)}.
\]
\end{definition}

Предлагаемое в этой работе определение слабо надёжного трудного бита основано
на существующем определении слабо надёжной односторонней функции. Поэтому для
начала вспомним её определение и его связь с определением обычных односторонних
функций из сложностной криптографии.

В сложностной криптографии существует определение односторонней функции, его
можно найти в книге Голдрейха \cite{goldreich}. Ниже приводится его неформальная
формулировка.

\begin{definition}[Из книги \cite{goldreich}]
Функция $f$ называется односторонней, если $f$ вычислима полиномиальным по
времени алгоритмом, но для любой схемы $A$, вероятность того, что $A$ корректно
обратит $f$ на входе длины $n$ "---
\[
\Pr_{x \gets \{0, 1\}^n}[(f \circ A \circ f)(x) = f(x)]
\]
должна убывать быстрее любого обратного полинома от $n$.
\end{definition}

Аналогично этому определению Хильтген \cite{hiltgen1993} предложил
следующее определение слабо надёжной односторонней перестановки порядка
$k$ (англ. \begin{foreignlanguage}{english}feebly-one-way of order
$k$\end{foreignlanguage}).

\begin{definition}[Хильтген \cite{hiltgen1993}]
Семейство перестановок $\{f_n\}$, где $f_n : \{0, 1\}^n \to \{0, 1\}^n$,
называется слабо односторонним порядка $k > 1$, если с ростом $n$ схемная
сложность $C(f_n^{-1})$ приближается к $k \cdot C(f_n)$:
\[
\begin{aligned}
&\lim_{l \to \infty} \sup_{n > l} C(f_n) = \infty
&
\text{и}&
&
\lim_{l \to \infty} \sup_{n > l} C(f_n^{-1}) / C(f_n) = k&
\end{aligned}
\]
\end{definition}

В этой дипломной работе предлагается определение слабой надёжности для другого
криптографического примитива из сложностной криптографии "--- трудного бита
односторонней перестановки.

Напомним, как определятся обычный трудный бит в сложностной криптографии.

\begin{definition}[Из книги \cite{goldreich}]
Предикат $h_n : \{0, 1\}^n \to \{0, 1\}$ называется трудным битом для
перестановки $f_n : \{0, 1\}^n \to \{0, 1\}^n$, если $h$ и $f$ вычислимы за
полиномиальное время, но никакая схема полниномиального размера $A$ не способна
эффективно вычислить $h(x)$ по заданному $f(x)$. То есть вероятность
\[
\Pr_{x \gets \{0, 1\}^n}[A(f_n(x)) = h_n(x)]
\]
убывает быстрее любого обратного полинома.
\end{definition}

Аналогично тому, как Хильтген получил определение слабо надёжной односторонней
функции из определения односторонней функции, предлагается следующее определение
для слабо надёжного трудного бита.

\begin{definition}[Слабо надёжный трудный бит]
Предикат $h_n : \{0, 1\}^n \to \{0, 1\}$ будем называть трудным битом порядка
$k$ с вероятностью $\alpha$ для перестановки $f_n : \{0, 1\}^n \to \{0, 1\}^n$ с
вероятностью $\alpha$ если
\[
\begin{aligned}
&\lim_{l \to \infty} \sup_{n > l} C(h_n) = \infty
&
\text{и}&
&
\lim_{l \to \infty} \sup_{n > l} \frac {C_\alpha(h_n \circ f_n^{-1})} {\max \{ C(h_n), C(f_n) \}} = k&
\end{aligned}
\]
\end{definition}

Интуинтивно такое определение можно понять так. Функция $h_n \circ f_n^{-1}$
здесь обозначает как раз задачу противника "--- по заданному $f_n(x)$ вычислить
$h_n(x)$. А функции $h_n$ и $f_n$ "--- это те задачи, которые решает
пользователь трудного бита; он может <<легко>> найти трудный бит $h_n(x)$ по
входу $x$, а также применить ко входу функцию $f_n$. И определение требует,
чтобы для решения задачи противника требовалось в $k$ раз больше операций, чем
для любой из этих двух решаемых пользователем задач. А точнее, мы требуем, чтобы
соотношение числа операций, совершаемых противником и пользователем,
приближалось к $k$ с ростом $n$.

\subsection{Обзор конструкций}

В этой дипломной работе приведены конструкции трудных битов порядка $(1
+ \frac 1 4)$ и $(1 + \frac 1 2)$ с вероятностью $7/8$. Эти конструкции
являются трудными битами для линейных односторонних функциях из работ Хильтгена
\cite{hiltgen1993} (с незначительными изменениями). Хильтген нашёл несколько
последовательностей квадратных обратимых матриц $\{M_n\}$ над $\F$, таких что
сложность умножения на такую матрицу имеет в $k$ раз меньшую схемную сложность,
чем сложность умножения на обратную к ней.

В этой работе мы будем рассматривать функции
\[
\begin{aligned}
&f_n : \F^n \times \F^n \to \F^n \times \F^n
&
\text{вида}&
&
f_n(x, y) = (x, M_n y).&
\end{aligned}
\]
В качестве семейства матриц
$\{M_n\}$ будут выбраны матрицы из работ Хильтгена, для которых уже доказаны
точные оценки на их схемную сложность и сложность обратных к ним. В качестве
трудного бита для $f_n$ мы рассмотрим внутреннее произведение \[
h_n(x, y) = x^\top y.
\]
В этом случае задача атакующего будет функцией $g_n = h_n \circ f_n^{-1}$:
\[
g_n(x, y) = x^\top M_n^{-1} y.
\]

Очевидная схема для вычисления $h_n$ будет оптимальной: $h_n(x, y) = \sum_i x_i
y_i$, для её реализации потребуется $n$ умножений и $n - 1$ сложений. Никакая
меньшая схема размера менее $2n - 1$ не сможет вычислить $h_n$ корректно, так
как $h_n$ имеет $2n$ входов и зависит от каждого из них. Поэтому $C(h_n) = 2n - 1$.

Сложность $f_n$ оценивается у Хильтгена, в конце этой работы мы просто
воспользуемся его результатами.

В следующих секциях мы определим семейства матриц $\{M_n^{-1}\}$ (вместе с ними
определятся и $M_n$, но начать нам удобнее с $M_n^{-1}$) и докажем нижнюю оценку
на сложность билинейных форм $g(x, y) = x^\top M_n^{-1} y$.

\section{Обозначения}

%% TODO
% Нужно написать о том, что я отождествляю {0, 1} и F₂

Символом $D_l$ обозначим квадратную матрицу $l \times l$, у которой на главной
диагонали и ниже неё стоят нули, а на всех остальных позициях "--- единицы.

\begin{definition}
Для последовательности $l_1 \dots l_k$ символом $A_{l_1, \dots l_k}$ обозначим
квадратную матрицу размера $(1 + \sum l_i) \times (1 + \sum l_i)$. В левом-верхнем
углу этой матрицы будет стоять единица. За ней вдоль главной диагонали будут идти
матрицы $D_{l_1}, D_{l_2} \dots D_{l_k}$ (как в блочно-диагональной матрице) до
правого нижнего угла. На всех остальных позициях, не занятых матрицами $D_{l_i}$
и левой верхней единицей, будут стоять единицы.
\end{definition}

Для удобства записи, мы будем иногда обозначать последовательность из нижнего
индекса $A_L$ вектором $L = (l_1 \dots l_k)$.

\begin{remark}
Строки и столбцы матрицы $A_L$ можно переставить таким образом, чтобы она
оказалась симметричной относительно главной диагонали.
\end{remark}

К примеру, $A_{1, 2, 3}$ будет выглядеть как
\[
\begin{bmatrix}
1 & 1 & 1 & 1 & 1 & 1 & 1 \\
1 & \mathbf{0} & 1 & 1 & 1 & 1 & 1 \\
1 & 1 & \mathbf{0} & \mathbf{1} & 1 & 1 & 1 \\
1 & 1 & \mathbf{0} & \mathbf{0} & 1 & 1 & 1 \\
1 & 1 & 1 & 1 & \mathbf{0} & \mathbf{1} & \mathbf{1} \\
1 & 1 & 1 & 1 & \mathbf{0} & \mathbf{0} & \mathbf{1} \\
1 & 1 & 1 & 1 & \mathbf{0} & \mathbf{0} & \mathbf{0} \\
\end{bmatrix},
\]
где жирным выделены подматрицы $D_1$, $D_2$, $D_3$.

Матрицы $A_L$ и будут использоваться в конструкциях в качестве $M_n^{-1}$, где
$n = 1 + \sum l_i$.

\section{Нижняя оценка}

В следующем утверждении используются такие обозначения:
\begin{itemize}
\item $\rowm_i(M)$ "--- функция, удаляющая $i$-тую строку из матрицы $M$;
\item $\colm_j(M)$ "--- аналогично, функция, удаляющая $j$-тый столбец из матрицы $M$;
\item $e_i$ "--- вектор, имеющий единицу на позиции $i$.
\end{itemize}

\begin{proposition} \label{easy}
Для любой строки $i$ (столбца) матрицы $A_L$ существует такой столбец $j$
(строка), что удаление этих строки и столбца из $A_L$ даст $A_{L - e_r}$ "---
другими словами, уменьшит на единицу одну из компонент $l_r$ вектора $L$:
\[
(\rowm_i \circ \colm_j)(A_L) = A_{L - e_r}.
\]

При этом, если после уменьшения компоненты $l_r$ на единицу она стала равна нулю,
то будем считать, что она совсем удаляется из вектора $L$.
\end{proposition}
\begin{proof}
Если $i = 1$ и была удалена первая строка, состоящая из всех единиц, тогда
установим $j := 2$ и удалим второй столбец. После этого появится новая состоящая
из единиц строка, и $l_1$ уменьшится на единицу.

Если $i > 1$ и была удалена строка, пересекающая какую-то из подматриц $D_r$,
тогда установим $j := i$ и удалим симметричный столбец. Легко проверить, что
\[
(\rowm_{i'} \circ \colm_{i'})(D_l) = D_{l-1}
\]
для любого $i' \leq l$.
\end{proof}

Доказательство нижней оценки, которое будет приведено в теореме
\ref{main}, пользуется техникой \begin{foreignlanguage}{english}gate
elimination\end{foreignlanguage} "--- мы фиксируем какие-то из входов функции
и для полученной функции применяем индукционное предположение. Но при
фиксировании каких-то входных бит квадратичной формы $(x, y) \mapsto x^\top A_L
y$ полученная функция может уже не быть квадратичной формой. Она будет иметь
вид $(x, y) \mapsto x^\top A_L y + u^\top x + v^\top y$ для каких-то векторов
$u$ и $v$. Поэтому оценка будет доказываться на функцию именно такого вида, и
некоторые вспомогательные утверждения ниже будут касаться функций такого вида.

\begin{proposition} \label{tough}
Для вектора $L$ обозначим $n = 1 + \sum l_i$.

Пусть $n \geq 2$, а функция
\[
g_n : \{0, 1\}^n \times \{0, 1\}^n \to \{0, 1\} \hspace{2.5cm}
g_n(x, y) = x^\top A_L y + u^\top x + v^\top y
\]
для некоторых $n$-мерных векторов $u$ и $v$. А $\xi : \{0, 1\}^2 \to \{0, 1\}$
"--- произвольная функция от двух бит, которая зависит от обоих своих входов.

Тогда никакая функция, корректно вычисляющая $g$ более чем на $7/8$ входов, не
является функцией от $\xi$ двух своих входных бит.
\end{proposition}

\begin{proof}
Пусть функция $g'(x, y)$ корректно вычисляет $g(x, y)$ на более, чем $7/8$ доле
всех $\{0, 1\}^{2n}$ входов.

Обозначим для удобства элементы матрицы $A_L$ как $\{a_{i, j}\}_{(i, j) \in {[n]^2}}$.

Рассмотрим два любых входных бита функции $g(x, y)$. Рассмотрим несколько
возможных случаев того, каким из входных векторов $x$ и $y$ функции $g$ они
принадлежат.
\begin{description}
\item{Биты "--- это $x_i$ и $x_j$.} У строк $i$ и $j$ матрицы $A_L$ обязательно
найдётся позиция (столбец), в которой они отличаются. Обозначим номер такого
столбца как $\delta$, а соответствующую столбцу переменную "--- как $y_\delta$.
Не теряя общности, будем считать, что в позиции $\delta$ строка $i$ имеет ноль,
а строка $j$ "--- единицу: $a_{i, \delta} = 0, a_{j, \delta} = 1$.

Ясно, что найдётся какая-то подстановка $\rho$, которая фиксирует все входные
переменные $g$, кроме $x_i, x_j, y_\delta$ так, что при этом $g' \rvert _ \rho$
корректно вычисляет $g \rvert _ \rho$ на более чем $7/8$ доле входов $\{0,
1\}^3$. Покажем, что $g' \rvert _ \rho$ не может быть функцией от $\xi$ двух своих
входных битов.

Поймём, какой вид будет иметь $g \rvert _ \rho$. 
\[
g(x, y) = \sum_{i', j'} a_{i', j'} x_{i'} y_{j'} + \sum_{i'} u_{i'} x_{i'} + \sum_{j'} v_{j'} y_{j'}.
\]
Если зафиксировать $\rho$, то получится (помним, что $a_{i, \delta} = 0, a_{j, \delta} = 1$)
\[
g \rvert _ \rho (x_i, x_j, y_\delta) = x_j y_\delta + x_i b_i + x_j b_j + y_\delta c_\delta + d,
\]
где $b_i, b_j, c_\delta$ и $d$ "--- это константы, которые получаются от
фиксирования входов $g$.

Уже сейчас можно понять, что такую функцию $g \rvert _ \rho$ нельзя вычислить на
более чем $7/8$ входов функцией от $\xi(x_i, x_j)$ ни при каких значениях констант
$b_i$, $b_j$, $c_\delta$ и $d$.

% Но убедимся в этом по-честному.
% Не теряя общности, можно считать, что константа $d = 0$.

\item{Биты "--- это $y_i$ и $y_j$.} Этот случай симметричен предыдущему.
\item{Биты "--- это $x_i$ и $y_j$.} Так как $A_L$ имеет одну строку и один
столбец, полностью состоящие из единиц, либо в строке $i$, либо в столбце $j$
найдётся единица на позиции, отличной от $(i, j)$. Не теряя общности будем
считать, что такая единица нашлась в столбце $j$ и расположена в клетке $(i',
j)$.

Аналогично предыдущему пункту выберем такую подстановку $\rho$ всех переменных,
кроме $x_i, x_{i'}, y_j$, чтобы $g \rvert _ \rho$ и $g' \rvert _ \rho$ совпадали
не менее, чем на $7/8$ входов.

Обозначив для краткости $a := a_{i, j}$, получим следующее выражение для $g
\rvert _ \rho$.
\[
g \rvert _ \rho (x_i, x_{i'}, y_j) = a x_i y_j + x_{i'} y_j + x_i b_i + x_{i'} b_{i'} + y_j c_j + d,
\]
где $b_{i}, b_{i'}, c_j$ и $d$ "--- константы.

Легко проверить вручную, что такая никакая функция, вычисляющая эту верно более
чем на $7/8$ доле входов, не может быть функцией от $\xi(x_i, y_j)$ ни при каких
значениях констант $b_{i}, b_{i'}, c_j$ и $d$.
\end{description}

(В том, что утвеждения из пунктов выше выполняются для $g \rvert _ \rho$, можно
убедиться вручную. У меня это получилось сделать только разбором случаев по
значениям констант. Либо можно перебрать программой все значения констант.)
\end{proof}

\begin{proposition} \label{single_one}
Пусть матрица $M$ имеет единицу в клетке $(i, j)$, а $u$ и $v$ "--- произвольные
вектора.

Тогда любая функция, вычисляющая $g(x, y) = x^\top A y + u^\top x + v^\top y$ на
$> 3/4$ доле входов, обязана зависеть от $x_i$ и от $y_j$.
\end{proposition}
\begin{proof}
Докажем, что $g$ обязана зависеть от $x_i$, а для $y_i$ доказательство будет
симметричным.

Пусть $g'(x, y)$ вычисляет $g(x, y)$ на $> 3/4$ доле входов. Ясно, что можно
зафиксировать все входные переменные этих функций, кроме $x_i$ и $y_j$, таким
образом, чтобы после фиксирования эти функции по-прежнему совпадали на $> 3/4$
доле входов. Обозначим такую подстановку как $\rho$.

Так как у функции $g \rvert _ \rho$ два входных бита: $x_i$ и $y_j$, которые
могут принимать всего $4$ значения, любая $g' \rvert _ \rho$, вычисляющая
её на $> 3/4$ доле входов обязана вычислять её коррекно на всех входах. Заметим,
что функция $g \rvert _ \rho(x_i, y_j)$ является многочленом с коэффициентом $1$
при $x_i y_j$:
\[
\begin{aligned}
&g \rvert _ \rho (x_i, y_j) = x_i y_j + a x_i + b y_j + c,
&
\text{где $a, b, c \in \{0, 1\}$.}&
\end{aligned}
\]
По этому выражению ясно, что никакая функция $g' \rvert _ \rho$, не зависящая от
$x_i$, не может вычислить $g \rvert _ \rho$ корректно на всех входах.

Значит, не существует не зависящей от $x_i$ функции $g'$, которая вычисляла бы
$g$ верно на $> 3/4$ доле входов.
\end{proof}

\begin{theorem} \label{main}
При $n = 1 + \sum l_i \geq 2$ для любых $u, v \in \{0, 1\}^n$ функция
\[
g_n(x, y) = x^\top A_L y + u^\top x + v^\top y
\]
не может быть вычислена на более чем $7/8$ входов схемой размера менее $3n -
\max \{l_i\} - 5$. То же самое другими словами:
\[
C_{7/8}(g_n) \geq 3n - \max \{l_i\} - 5.
\]
\end{theorem}
\begin{proof}
Индукция по $n$.

\emph{База $n = 2$}. Очевидно, $C_{7/8}(g_n) \geq 0$.

\emph{Переход $n > 2$.} Рассмотрим любую функцию $g'_n$, которая корректно
вычисляет $g_n$ на более чем $7/8$ доле входов. Рассмотрим схему минимального
размера для $g'_n$. Обозначим первый гейт этой схемы в порядке топологической
сортировки как $\xi$. На вход $\xi$ подаются два входа схемы. По предложению \ref{tough}
один из этих входов должен подаваться на какой-то ещё гейт схемы. Зафиксируем
этот вход схемы $z_1$ в значение $b_1$, удалив из схемы два гейта. Если $g'_n$
вычисляла $g_n$ на $> 7/8$ доле входов, то можно выбрать $b_1$ так, чтобы
$g'_n \rvert_{z_1 = b_1}$ вычисляла $g_n \rvert_{z_1 = b_1}$ на $ > 7/8$ доле входов.

Такое фиксирование удалит либо одну строку, либо столбец из матрицы $A_L$. Не
теряя общности, будем считать, что была удалена строка. Обозначим её номер как
$i$. По предложению \ref{easy} найдётся столбец $j$, такой что после удаления
строки $i$ и столбца $j$ из матрицы $A_L$ получится $A_{L'}$, где $L' = L - e_r$
для некоторого $r$.

Для удаления столбца $j$ из матрицы, полученной после фиксирования $z_1$,
зафиксируем соответствующий ему входной бит. Обозначим этот бит как $z_2$, а
значение, в которое его зафиксируем, как $b_2$. Значение $b_2$ можно выбрать
таким образом, чтобы $g'_n \rvert _ {z_1 = b_1, z_2 = b_2}$ верно вычисляла
$g_n \rvert _ {z_1 = b_1, z_2 = b_2}$ на $> 7/8$ входов.

Напомним, что $L = (l_1, \dots l_k)$. Рассмотрим два случая.
\begin{itemize}
\item Если $k \geq 2$, то после удаления строки $i$ в каждом столбце останется
как минимум одна единица. Поэтому из предложения \ref{single_one} следует, что
$g' \rvert _ {z_1 = b_1}$ зависит от $z_2$. А это значит, что при фиксировании
$z_2 = b_2$ из схемы удалится как минимум один гейт. В этом случае, мы удалили
три гейта из схемы, поэтому $C(g' \rvert _ {z_1 = b_1, z_2 = b_2}) \leq C(g') - 3$
и уменьшили одно из значений $l_i$ на единицу. По индукционному предположению
\[
\begin{aligned}
C(g' \rvert _ {z_1 = b_1, z_2 = b_2}) &\geq 3(n - 1) - \underbrace{\max \{l_i'\}}_{\leq \max \{l_i\}} - 5 \\
&\geq 3(n - 1) - \max \{l_i\} - 5.
\end{aligned}
\]
Отсюда получаем требуемое утверждение:
\[
\begin{aligned}
C_{7/8}(g) &\geq C(g' \rvert _ {z_1 = b_1, z_2 = b_2}) + 3 \\
           &\geq 3n - \max \{l_i\} - 5.
\end{aligned}
\]

%% TODO: Заменить C на C_{7/8} в этом доказательстве

\item Если $k = 1$. В этом случае после фиксирования $z_2 = b_2$ из схемы могло
ничего и не удалиться. Но всё равно мы удалили из схемы как минимум два гейта
фиксированием $z_1 = b_1$. Поэтому аналогично предыдущему пункту верно $C(g'
\rvert _ {z_1 = b_1, z_2 = b_2}) \leq C(g') - 2$, к тому же по индукционному
предложению
\[
\begin{aligned}
C(g' \rvert _ {z_1 = b_1, z_2 = b_2}) &\geq 3(n - 1) - \underbrace{\max \{l_i'\}}_{\max \{l_i\} - 1} - 5 \\
&= 3(n - 1) - \max \{l_i\} + 1 - 5.
\end{aligned}
\]
Получаем требуемое утверждение:
\[
\begin{aligned}
C_{7/8}(g) &\geq C(g' \rvert _ {z_1 = b_1, z_2 = b_2}) + 2 \\
           &\geq 3n - \max \{l_i\} - 5.
\end{aligned}
\]
\end{itemize}
\end{proof}

\section{Следствия и конструкции}
Обозначим символом $M_{n, t}$ матрицу $A_L$, где вектор $L = \{l_1 \dots l_t\}$.
Сумма компонент $L$ равна $n - 1 = \sum l_i$, а значения компонент отличаются
максимум на единицу: $l_i \in \left \{ \floor{\frac {n - 1} t}, \ceil{\frac {n -
1} t} \right\}$.

Следующее является следствием теоремы \ref{main}.

\begin{corollary}
Для $g_{n, t}(x, y) = x^\top M_{n, t} y$ верно
\[
C_{7/8}(g_{n, t}) \geq 3n - \ceil*{\frac {n - 1} t} - 5.
\]
\end{corollary}

% TODO: Показать здесь M^{-1}_{n, t}

В работе Хильтгена \cite[определение 8]{hiltgen1993} построена такая
односторонняя функция $\lambda_{n, t}$ и доказана следующая теорема о её
сложности.

\begin{theorem}[Хильтген]
Пусть $\lambda_{n, t}(x) = M_{n, t}^{-1} x$, а $\lambda^{-1}_{n, t}(x) = M_{n,
t} x$, тогда
\[
n \leq C(\lambda_{n, t}) \leq n + t - 1 \quad \quad C(\lambda_{n, t}^{-1}) = \floor*{ \frac {2t - 1} t (n - 1) }
\]
\end{theorem}

Теперь можно выбрать в качестве односторонней функции
\[
f_{n,t}(x, y) = (x, \lambda_{n, t}(y)),
\]
а в качестве трудного бита для неё скалярное произведение $h_n(x, y) = x^\top y$.
Тогда функция
\[
g_{n, t}(x, y) = (h \circ f_{n, t}^{-1})(x, y) = x^\top M_{n, t} y
\]
окажется как раз той функцией, которую
должен вычислить атакующий, которому известно значение односторонней функции
$f_{n, t}(x, y)$ и который хочет вычислить $h_n(x, y)$.

Теперь, выбирая конкретные значения для $t$, построим несколько односторонних
функций с трудным битом. (Помним, что $C(h_n) = 2n - 1$ и не зависит от $t$.)
\begin{description}
\item[$t = 2$.] В этом случае получается слабо надёжная односторонная функция
порядка $3/2$ из работы \cite{hiltgen1993} и слабо надёжный трудный бит для
неё порядка $(3 - 1/2) / 2 = (1 + 1/4)$ с вероятностью $7/8$.
\[
\begin{aligned}
C(f_{n, t}) &\leq n + 1 \\
C(f_{n, t}^{-1}) &\geq \frac 3 2 (n - 1) \\
C(g_{n, t}) &\geq (3 - \frac 1 2)n - 6. \\
\end{aligned}
\]
\item[$t = n - 1$.] Это крайний случай, здеь получается, что $f_{n, t}$ так же
легко обратить, как и вычислить, но оценка на $C(g)$ получается самая лучшая:
\[
\begin{aligned}
C(f_{n, t}) &\leq 2n - 2 \\
C(f_{n, t}^{-1}) &\geq 2n - 3 \\
C(g_{n, t}) &\geq 3n - 6. \\
\end{aligned}
\]
$f_{n,t}$ "--- слабо надёжная односторонняя функция порядка $1$, а $h_{n}$ "---
трудный бит для неё порядка $1 + 1/2$ с вероятностью $7/8$.
\item[$t = \sqrt{n}$.] В этом случае получаются асимптотически наилучшие
соотношения между сложностями функций:
\[
\begin{aligned}
C(f_{n, t}) &\leq n + \sqrt{n} - 1 \\
C(f_{n, t}^{-1}) &\geq 2n - \sqrt n - 3 \\
C(g_{n, t}) &\geq 3n - \sqrt{n} - 6. \\
\end{aligned}
\]
Функция $f_{n, t}$ является слабо односторонней порядка $2$, а $h_{n}$
является трудным битом для неё поряка $1 + 1/2$ с вероятностью $7/8$.
\end{description}

\bibliography{main}{}
\bibliographystyle{plain}

\end{document}
