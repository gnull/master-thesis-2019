\documentclass[oneside, a4paper]{article}

\usepackage[T1,T2A]{fontenc}
\usepackage[utf8]{inputenc}
\usepackage[english,russian]{babel}
\usepackage{url, hyperref}
\usepackage[shortlabels]{enumitem}
\usepackage{amssymb,amsthm,amsmath,mathtools}
\usepackage{listings}

\usepackage{geometry}

% \geometry{
%    a4paper,
%    left=20mm,
%    right=20mm,
%  }

\newtheorem{theorem}{Теорема}
\newtheorem{proposition}{Предложение}
\newtheorem{lemma}{Лемма}

\renewcommand{\thesection}{}
\renewcommand{\thesubsection}{}

\DeclareMathOperator{\lcm}{LCM}
\DeclareMathOperator*{\argmax}{arg\,max}

\DeclarePairedDelimiter\abs{\lvert}{\rvert}
\DeclarePairedDelimiter\ang{\langle}{\rangle}

\DeclareRobustCommand{\divby}{%
  \mathrel{\vbox{\baselineskip.65ex\lineskiplimit0pt\hbox{.}\hbox{.}\hbox{.}}}%
}

\begin{document}

\title{}
\author{Олейников Иван}
\date{\today}
\maketitle

Определим матрицу $C_n$ размера $n \times n$ так: она имеет нули на всех
позициях главной диагонали, кроме самой нижней, и единицы на всех остальных
позициях.

\begin{proposition} \label{easy}
Для любой строки $i$ (столбца) матрицы $C_n$ существует такой столбец $j$
(строка), что удаление этих строки и столбца из $C_n$ даст $C_{n-1}$.
\end{proposition}
\begin{proof}
Если $i \neq n$, то можно положить $j = i$. Если $i = n$, то установим $j = n - 1$.
\end{proof}

\begin{proposition} \label{tough}
Если функция $f_n(x, y) = x^\top C_n y + u^\top x + v^\top y$ для некоторых
$n$-мерных векторов $u$ и $v$, а $g : \{0, 1\}^2 \to \{0, 1\}$ "---
произвольная функция от двух бит, которая зависит от них обоих, то никакая функция,
корректно вычисляющая $f$
более чем на $7/8$ входов, не является функцией от $g$ двух своих входных бит.
\end{proposition}

Доказательство этого предложения "--- самая рутинная часть этого документа. Поэтому
сначала мы применим его в следующей лемме, а затем докажем его.

\begin{lemma}
При $n \geq 2$ для любых $u, v \in \{0, 1\}^n$ функция $f_n(x, y) = x^\top C_n y + u^\top x + v^\top y$
не может быть вычислена на более чем $7/8$ входов схемой размера менее $3n - 5$. То же
самое другими словами:
\[
C_{7/8}(f_n) \geq 3n - 6.
\]
\end{lemma}
\begin{proof}
Индукция по $n$.

\emph{База $n = 2$}. Очевидно, $C_{7/8}(f_n) \geq 0$.

\emph{Переход $n > 2$.} Рассмотрим любую функцию $h_n$, которая корректно
вычисляет $f_n$ на более чем $7/8$ доле входов. Рассмотрим схему минимального
размера для $h_n$. Обозначим первый гейт этой схемы в порядке топологической
сортировки как $g$. На вход $g$ подаются два входа схемы. По предложению \ref{tough}
один из этих входов должен подаваться на какой-то ещё гейт схемы. Зафиксируем
этот вход схемы $z_1$ в значение $b_1$, удалив из схемы два гейта. Если $h_n$
вычисляла $f_n$ на $7/8$ доле входов, то можно выбрать $b_1$ так, чтобы
$h_n \rvert_{z_1 = b_1}$ вычисляла $f_n \rvert_{z_1 = b_1}$ на $7/8$ доле входов.

Такое фиксирование удалит либо одну строку, либо столбец из матрицы $C_n$. Не
теряя общности, будем считать, что была удалена строка. Обозначим её номер как
$i$. По предложению \ref{easy} найдётся столбец $j$, такой что после удаления
строки $i$ и столбца $j$ из матрицы $C_n$ получится $C_{n-1}$.

Для удаления столбца $j$ из матрицы, полученной после фиксирования $z_1$,
зафиксируем соответствующий ему входной бит. Обозначим этот бит как $z_2$, а
значение, в которое его зафиксируем, как $b_2$. Значение $b_2$ можно выбрать
таким образом, чтобы $h_n \rvert _ {z_1 = b_1, z_2 = b_2}$ верно вычисляла
$f_n \rvert _ {z_1 = b_1, z_2 = b_2}$ на $7/8$ входов. При фиксироавании
$z_2 = b_2$ из схемы для $h_n \rvert _ {z_1 = b_1}$ удалится как минимум один
гейт, ведь если бы $h_n \rvert _ {z_1 = b_1}$ не зависела от $z_2$, она бы не
могла корректно вычислять $f_n \rvert _ {z_1 = b_1}$ на $7/8$ входов.

Мы зафиксировали два входа $f_n$ и $h_n$, удалив два гейта из схемы для $h_n$.
При этом из $C_n$ удалились строка и столбец, а полученная после этого матрица
равна $C_{n-1}$. По индукционному предположению $C_{7/8}(h_n \rvert _ {z_1 = b_1,
z_2 = b_2}) \geq 3(n-1) - 6$. Отсюда получаем, что $C_{7/8}(h_n) \geq 3n - 6$.
\end{proof}

Теперь докажем оставшееся предложение \ref{tough}.

\begin{proof}
Рассмотрим два любых входных бита функции $f(x, y)$. Рассмотрим несколько
возможных случаев того, каким из входных векторов $x$ и $y$ функции $f$ они
принадлежат.
\begin{description}
\item{Биты "--- это $x_1$ и $x_2$.} \dots
\item{Биты "--- это $y_1$ и $y_2$.} Этот случай симметричен предыдущему.
\item{Биты "--- это $x_1$ и $y_1$.} \dots
\end{description}
\end{proof}

\end{document}
