\usepackage{fontspec} % loaded by polyglossia, but included here for transparency
\usepackage{polyglossia}
\setmainlanguage{russian}
\setotherlanguage{english}
\setkeys{russian}{babelshorthands=true}

\defaultfontfeatures{Ligatures=TeX}
\setmainfont{FreeSerif}
\newfontfamily\cyrillicfont{FreeSerif}
\setsansfont{Liberation Sans}
\setmonofont{FreeMono}

\usepackage{url, hyperref}
\usepackage[shortlabels]{enumitem}
\usepackage{amssymb,amsthm,amsmath,mathtools,thmtools,thm-restate}
\usepackage{listings}

\usepackage{graphicx}
\usepackage{caption}
\usepackage{subcaption}

\usepackage{savesym}
\savesymbol{C}
\savesymbol{G}
\usepackage{complexity}
\restoresymbol{TXF}{C}
\restoresymbol{TXF}{G}

\usepackage{tikz}

\usepackage[nottoc]{tocbibind}

\usepackage{geometry}

\geometry{
  a4paper,
  left=30mm,
  right=15mm,
  top=20mm,
  bottom=20mm,
  includefoot,
}

\usepackage{pdfpages}

\newtheorem{theorem}{Теорема}
\newtheorem{exercise}{Упражнение}
\newtheorem{corollary}{Следствие}
\newtheorem{lemma}{Лемма}
\theoremstyle{definition}
\newtheorem{definition}{Определение}
\newtheorem{construction}{Конструкция}
\theoremstyle{remark}
\newtheorem{remark}{Замечание}

\newcommand\rowm{\ensuremath{\operatorname{row}^-}}
\newcommand\colm{\ensuremath{\operatorname{col}^-}}
\newcommand\F{\ensuremath{{\mathbb F}_2}}

\DeclareMathOperator{\lcm}{LCM}
\DeclareMathOperator*{\argmax}{arg\,max}
\DeclareMathOperator*{\val}{val}

\DeclarePairedDelimiter\abs{\lvert}{\rvert}
\DeclarePairedDelimiter\ang{\langle}{\rangle}
\DeclarePairedDelimiter\floor{\lfloor}{\rfloor}
\DeclarePairedDelimiter\ceil{\lceil}{\rceil}

\DeclareRobustCommand{\divby}{%
  \mathrel{\vbox{\baselineskip.65ex\lineskiplimit0pt\hbox{.}\hbox{.}\hbox{.}}}%
}

\setcounter{tocdepth}{1}

\hypersetup{
    colorlinks, urlcolor={black}, % Все ссылки черного цвета, кликабельные
    linkcolor={black}, citecolor={black}, filecolor={black},
    pdfauthor={Иван Олейников},
    pdftitle={Доказуемо надёжные в слабом смысле трудные биты}
}

\renewcommand{\baselinestretch}{1.6} % Полуторный межстрочный интервал
\parindent 1.25cm % Абзацный отступ

\setlist[enumerate,itemize]{leftmargin=12.7mm}

\RequirePackage{caption}
\DeclareCaptionLabelSeparator{defffis}{ -- } % Разделитель
\captionsetup[figure]{justification=centering, labelsep=defffis, format=plain} % Подпись рисунка по центру
\captionsetup[table]{justification=justified, labelsep=defffis, format=plain, singlelinecheck=false} % Подпись таблицы слева
\addto\captionsrussian{\renewcommand{\figurename}{Рисунок}} % Имя фигуры

\usepackage[uppercase,center]{titlesec}

\titlespacing\section{0pt}{12pt plus 4pt minus 2pt}{1cm}
\titlespacing\subsection{0pt}{12pt plus 4pt minus 2pt}{0.5cm}
