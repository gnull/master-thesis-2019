\documentclass[mathserif,serif]{beamer} % Add handout here to disable animations
\usetheme{Boadilla}

\usepackage{fontspec} % loaded by polyglossia, but included here for transparency
\usepackage{polyglossia}
\setmainlanguage{russian}
\setotherlanguage{english}
\setkeys{russian}{babelshorthands=true}

\defaultfontfeatures{Ligatures=TeX}
\setmainfont{CMU Serif}
\newfontfamily\cyrillicfont{CMU Serif}
\setsansfont{CMU Sans Serif}
\setmonofont{FreeMono}

\usepackage{savesym}
\savesymbol{C}
\savesymbol{G}
\usepackage{complexity}
\restoresymbol{TXF}{C}
\restoresymbol{TXF}{G}

\usepackage{amssymb,amsthm,amsmath,mathtools,thmtools,thm-restate}

\newtheorem{thm}{Теорема}
\newtheorem{exercise}{Упражнение}
\newtheorem{corr}{Следствие}
\newtheorem{lemm}{Лемма}
\theoremstyle{definition}
\newtheorem{defn}{Определение}
\newtheorem{construction}{Конструкция}
\theoremstyle{remark}
\newtheorem{remark}{Замечание}

\newcommand\F{\ensuremath{{\mathbb F}_2}}
\DeclarePairedDelimiter\floor{\lfloor}{\rfloor}
\DeclarePairedDelimiter\ceil{\lceil}{\rceil}

\title{Доказуемо надёжные в слабом смысле трудные биты}
\subtitle{Магистерская диссертация}
\author{Олейников Иван Викторович}
\date{11 июня 2019}
\subject{Computer Science}

\begin{document}

\begin{frame}
\vspace{2cm}
\begin{center}
  \textbf{Доказуемо надёжные в слабом смысле трудные биты}
  \\
  Магистерская диссертация
  \\[1em]
  {
    Олейников Иван Викторович
  }
  \\[2em]
\end{center}

\hfill
\begin{minipage}{0.45\textwidth}
\textbf{Научный руководитель:} \\
доктор физико"=математических наук, доцент \\
Гирш Эдуард Алексеевич
\end{minipage}

\begin{center}
\mbox{}
\vfill
\small Санкт-Петербург, 2019
\end{center}
\end{frame}

\begin{frame}
  \frametitle{Мотивация для слабых определений надёжности}
  Мы не умеем доказывать надёжность (почти) ни для одного криптографического
  примитива.
  \begin{itemize}
  \item Это трудно, ведь если доказать существование, скажем, односторонней
  функции то из этого последует $\P \neq \NP$.
  \item Обычно такая надёжность требует, чтобы задачи честных пользователей
  решались за полиномиальное число операций, а задача противника (взлом) "---
  нет.
  \end{itemize}
  \pause
  Насколько нужно ослабить требования надёжности, чтобы получилось что-то, что
  мы умеем доказывать?
  \begin{itemize}
  \item Надёжность в слабом смысле порядка $k$ требует, чтобы для взлома
  требовалось в $k$ раз больше операций, чем для решения задачи честного
  пользователя.
  \end{itemize}
\end{frame}

\begin{frame}
  \frametitle{Существующие конструкции}
  Существующие работы по надёжным в слабом смысле конструкциям:
  \begin{itemize}
  \item Односторонние функции "--- Хильген 1993 г.
  \item Функции с секретом (\begin{english}trapdoor function\end{english}) "---
  Гирш, Николенко, Меланич, Давыдов 2009-2012 г.
  \end{itemize}
  \pause
  В этой работе строится:
  \begin{itemize}
  \item Функция с трудным битом.
  \end{itemize}
\end{frame}

\begin{frame}
  \frametitle{Функция и её трудный бит}
  {
    \centering
    \includegraphics[width=0.5\textwidth]{../standalone/figure_hp}
    \[
    \begin{aligned}
    f : \{0, 1\}^n \to \{0, 1\}^n
    & \quad &
    h : \{0, 1\}^n \to \{0, 1\}
    \end{aligned}
    \]
  }
  \begin{defn}
  Предикат $h$ является трудным битом для перестановки $f$ порядка $k$ с
  вероятностью $\alpha$, если
  \[
  \lim_{l \to \infty} \sup_{n > l} \frac {C_\alpha(h_n \circ f_n^{-1})} {\max \{ C(h_n), C(f_n) \}} = k, \quad
  \parbox{5cm}{ \footnotesize
     где $C_\alpha(\cdot)$ "--- схемная сложность вычисления функции на $> \alpha$ доле входов, а $C(\cdot)$ "--- на всех входах.
  }
  \]
  \end{defn}
\end{frame}

\begin{frame}
  \frametitle{Обзор конструкции}
  Хильтген построил линейную перестановку $\lambda_{n, s} : \F^n \to \F^n$ и
  доказал точные оценки на сложности $C(\lambda_{n, s})$ и $C(\lambda_{n, s}^{-1})$.
  \[
  \lambda_{n, s}(x) = \Lambda_{n, s} x
  \]
  \pause
  Зададим
  \[
  f_{n, s}(x, y) = (x, \lambda_{n, s}(y)) = (x, \Lambda_{n, s} y).
  \]
  \pause
  Трудным битом для $f_{n ,s}$ будет скалярное произведение:
  \[
  h_n (x, y) = x^\top y.
  \]
\end{frame}

\begin{frame}
\frametitle{План доказательства}
\fbox{
  \begin{minipage}{\textwidth}
    Напоминание с прошлого слайда:
    \[
    \begin{aligned}
    f_{n, s}(x, y) &= (x, \lambda_{n, s}(y)) = (x, \Lambda_{n, s} y) \\
    h_n (x, y) &= x^\top y. \\
    \end{aligned}
    \]
  \end{minipage}
}
\pause
\\[1em]
Чтобы оценить снизу
\[
  \lim_{l \to \infty} \sup_{n > l} \frac {C_\alpha(h_n \circ f_n^{-1})} {\max \{ C(h_n), C(f_n) \}},
\]
нужны оценки
\begin{tabular}{l | l}
сверху на $C(h_n)$                      & \onslide<3->{очевидно $C(h_n) = 2n - 1$} \\
сверху на $C(f_{n, s})$                 & \onslide<4->{очевидно $C(f_{n, s}) = C(\lambda_{n, s})$} \\
снизу на $C_\alpha(h_n \circ f_{n, s}^{-1})$ & \onslide<5->{доказывается в этой работе.} \\
\end{tabular}
\end{frame}

\begin{frame}
\frametitle{Нижняя оценка}
\fbox{
  \begin{minipage}{\textwidth}
    Напоминание с прошлого слайда:
    \[
    \begin{aligned}
    f_{n, s}(x, y) &= (x, \lambda_{n, s}(y)) = (x, \Lambda_{n, s} y) \\
    h_n (x, y) &= x^\top y. \\
    \end{aligned}
    \]
  \end{minipage}
}
\pause
\\[1em]
Функция $h_n \circ f_{n, s}^{-1}$ является билинейной формой:
\[
(h_n \circ f_{n, s}^{-1})(x, y) = h_n(x, \Lambda_{n, s}^{-1}y) = x^\top \Lambda_{n, s}^{-1} y.
\]
\pause
\begin{thm}
\[
C_{7/8}(h_n \circ f_{n, s}^{-1}) \geq 3n - \ceil{n / s} - 5.
\]
\end{thm}
\end{frame}

\begin{frame}
  \frametitle{Заключение}
  {
    \centering
    \includegraphics[width=0.5\textwidth]{../standalone/figure_hp}
    \[
    \begin{aligned}
    f : \{0, 1\}^n \to \{0, 1\}^n
    & \quad &
    h : \{0, 1\}^n \to \{0, 1\}
    \end{aligned}
    \]
  }
  \begin{construction}[$s = \sqrt{n}$.]
  Получается трудный бит порядка $1 + 1/2$ с вероятностью $7/8$.
  \[
  C_{7/8}(g_{n, s}) \geq 3n - \sqrt{n} - 6.
  \]
  Функцию $f_{n, s}$ обратить в $2$ раза труднее, чем вычислить.
  \end{construction}
\end{frame}

\end{document}
