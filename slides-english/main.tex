\documentclass[mathserif,serif]{beamer} % Add handout here to disable animations
\usetheme{Boadilla}

\usepackage{fontspec} % loaded by polyglossia, but included here for transparency
\usepackage{polyglossia}
\setmainlanguage{english}

\defaultfontfeatures{Ligatures=TeX}
\setmainfont{CMU Serif}
\setsansfont{CMU Sans Serif}
\setmonofont{FreeMono}

\usepackage{savesym}
\savesymbol{C}
\savesymbol{G}
\usepackage{complexity}
\restoresymbol{TXF}{C}
\restoresymbol{TXF}{G}

\usepackage{amssymb,amsthm,amsmath,mathtools,thmtools,thm-restate}

\newtheorem{thm}{Theorem}
\theoremstyle{definition}
\newtheorem{defn}{Definition}
\newtheorem{construction}{Contruction}

\newcommand\F{\ensuremath{{\mathbb F}_2}}
\DeclarePairedDelimiter\floor{\lfloor}{\rfloor}
\DeclarePairedDelimiter\ceil{\lceil}{\rceil}

\title{Feebly secure hard-core predicates}
\subtitle{(Master thesis)}
\author{Ivan Oleynikov\thanks{St. Petersburg Academic University} \and Advisor: Edward A. Hirsch\thanks{St. Petersburg Steklov Mathematical Institute}}
\date{\today}
\subject{Computer Science}

\begin{document}

\frame{\titlepage}

\begin{frame}
  \frametitle{Motivation for feebly secure contructions}
  We have (almost) no provably secure cryptographic contructions.
  \begin{itemize}
  \item Finding such constructions is hard, since it would imply $\P \neq \NP$.
  \item Typically, for a contruction to be secure we require that the tasks of
  honest parties can be solved in polynomial time, but the tasks of an adversary
  (i.e. breaking the construction) can't.
  \end{itemize}
  \pause
  Maybe we could weaken the security requirements to obtain something we can
  prove?
  \begin{itemize}
  \item Feeble security of order $k$ requires adversary to perform at least $k$
  times more operations than honest parties do.
  \end{itemize}
\end{frame}

\begin{frame}
  \frametitle{Existing constructions}
  Existing feebly secure constructions:
  \begin{itemize}
  \item One-way functions --- Hiltgen in 1993.
  \item Trapdoor functions --- Hirsch, Nikolenko, Melanich, Davydow in 2009-2012.
  \end{itemize}
  \pause
  In this work we construct:
  \begin{itemize}
  \item A feebly secure hard-core predicate for a function.
  \end{itemize}
\end{frame}

\begin{frame}
  \frametitle{Hard-core predicate of a function}
  {
    \centering
    \includegraphics[width=0.5\textwidth]{../standalone/figure_hp}
    \[
    \begin{aligned}
    f : \{0, 1\}^n \to \{0, 1\}^n
    & \quad &
    h : \{0, 1\}^n \to \{0, 1\}
    \pause
    \end{aligned}
    \]
  }
  \begin{defn}
  Predicate $h$ is a hard-core predicate for permutation $f$ of order $k$
  with probability $\alpha$, iff
  \[
  \lim_{l \to \infty} \sup_{n > l} \frac {C_\alpha(h_n \circ f_n^{-1})} {\max \{ C(h_n), C(f_n) \}} = k, \quad
  \parbox{5cm}{ \footnotesize
     where $C_\alpha(\cdot)$ is the circuit complexity of computing a function
     on $> \alpha$ fraction of inputs, and $C(\cdot)$ --- for computing it on
     all inputs.
  }
  \]
  \end{defn}
\end{frame}

\begin{frame}
  \frametitle{Overview of the Contruction}
  Hiltgen has presented a linear permutation $\lambda_{n, s} : \F^n \to \F^n$ and
  found the exact values of the complexities $C(\lambda_{n, s})$ и $C(\lambda_{n,
  s}^{-1})$.
  \[
  \lambda_{n, s}(y) = \Lambda_{n, s} y
  \]
  \pause
  Define
  \[
  f_{n, s}(x, y) = (x, \lambda_{n, s}(y)) = (x, \Lambda_{n, s} y).
  \]
  \pause
  We prove that the inner (scalar) product
  \[
  h_n (x, y) = x^\top y
  \]
  is a hard-core predicate for permutation $f_{n ,s}$
\end{frame}

\begin{frame}
\frametitle{Overview of the Proof}
\fbox{
  \begin{minipage}{\textwidth}
    Reminder from the previous slide:
    \[
    \begin{aligned}
    f_{n, s}(x, y) &= (x, \lambda_{n, s}(y)) = (x, \Lambda_{n, s} y) \\
    h_n (x, y) &= x^\top y. \\
    \end{aligned}
    \]
  \end{minipage}
}
\pause
\\[1em]
To obtain a lower bound on
\[
  \lim_{l \to \infty} \sup_{n > l} \frac {C_\alpha(h_n \circ f_n^{-1})} {\max \{ C(h_n), C(f_n) \}},
\]
we need the following bounds:
\begin{tabular}{l | l}
upper bound on $C(h_n)$                      & \onslide<3->{trivially $C(h_n) = 2n - 1$} \\
upper bound on $C(f_{n, s})$                 & \onslide<4->{trivially $C(f_{n, s}) = C(\lambda_{n, s})$} \\
lower bound on $C_\alpha(h_n \circ f_{n, s}^{-1})$ & \onslide<5->{proven in this work.} \\
\end{tabular}
\end{frame}

\begin{frame}
\frametitle{The lower bound}
\fbox{
  \begin{minipage}{\textwidth}
    Reminder from the previous slide:
    \[
    \begin{aligned}
    f_{n, s}(x, y) &= (x, \lambda_{n, s}(y)) = (x, \Lambda_{n, s} y) \\
    h_n (x, y) &= x^\top y. \\
    \end{aligned}
    \]
  \end{minipage}
}
\pause
\\[1em]
Function $h_n \circ f_{n, s}^{-1}$ is a bilinear form:
\[
(h_n \circ f_{n, s}^{-1})(x, y) = h_n(x, \Lambda_{n, s}^{-1}y) = x^\top \Lambda_{n, s}^{-1} y.
\]
\pause
\begin{thm}
\[
C_{7/8}(h_n \circ f_{n, s}^{-1}) \geq 3n - \ceil{n / s} - 5.
\]
\end{thm}
\end{frame}

\begin{frame}
  \frametitle{Conclusion}
  {
    \centering
    \includegraphics[width=0.5\textwidth]{../standalone/figure_hp}
    \[
    \begin{aligned}
    f : \{0, 1\}^n \to \{0, 1\}^n
    & \quad &
    h : \{0, 1\}^n \to \{0, 1\}
    \end{aligned}
    \]
  }
  \begin{construction}[$s = \sqrt{n}$.]
  Predicate $h_n$ is a hard-core predicate for permutation $f_{n, s}$ of order
  $3/2 = 1 + 1/2$ with probability $7/8$.
  \[
  \begin{aligned}
  C(h_n) &= 2n - 1 & C_{7/8}(g_{n, s}) &\geq 3n - \sqrt{n} - 6 \\
  C(f_n) &= n + \sqrt{n} - 1 \\
  \end{aligned}
  \]
  \end{construction}
\end{frame}

\end{document}
